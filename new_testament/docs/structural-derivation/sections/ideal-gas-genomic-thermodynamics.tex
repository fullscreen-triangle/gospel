\section{Ideal Gas Laws for Genomic Systems}

\subsection{Triple Equivalence Applied to Genomics}

\begin{theorem}[Genomic Triple Equivalence]
\label{thm:genomic_triple}
Genomic systems exhibit triple equivalence:
\begin{equation}
\text{Charge Oscillation} \equiv \text{Categorical Gene State} \equiv \text{Sequence Partition}
\end{equation}
\end{theorem}

\begin{proof}
\textbf{Oscillation}: DNA charge oscillates at metabolic frequencies ($\omega \sim 0.1$--100 s$^{-1}$) through ion concentration fluctuations, ATP synthesis cycles, and membrane potential oscillations.

\textbf{Category}: Gene expression states partition into discrete categories (off, low, medium, high expression) distinguishable through transcriptional measurements.

\textbf{Partition}: Genomic sequence partitions into functional domains (genes, regulatory elements, intergenic regions) separated by boundaries.

These three perspectives describe identical structure: charge oscillations create partition boundaries (chromatin domains), which define categorical states (gene expression levels). The equivalence follows from bounded phase space (nuclear volume) and categorical observation (transcriptional measurements).
\end{proof}

\subsection{Fundamental Identity for Genomic Oscillators}

\begin{theorem}[Genomic Oscillator Identity]
\label{thm:genomic_identity}
For genomic oscillator with mass $M$ (effective charge mass), frequency $\omega$, and partition timescale $\tau_p$:
\begin{equation}
\frac{dM}{dt} = \frac{\omega}{2\pi/M} = \frac{1}{\langle\tau_p\rangle}
\end{equation}
\end{theorem}

\begin{proof}
\textbf{Oscillatory form}: Charge oscillation frequency $\omega = 2\pi/T$ where $T$ is period. Mass flow rate: $dM/dt = M/T = M\omega/(2\pi)$.

\textbf{Categorical form}: Categorical distinction rate equals inverse mean partition dwell time: $1/\langle\tau_p\rangle$.

\textbf{Partition form}: Partition crossing rate equals frequency divided by angular period: $\omega/(2\pi/M)$.

Triple equivalence establishes identity of three forms.
\end{proof}

\subsection{Temperature as Categorical Actualization Rate}

\begin{definition}[Genomic Temperature]
\label{def:genomic_temperature}
Temperature quantifies rate of categorical state actualization:
\begin{equation}
T_{\text{genomic}} = \frac{\langle E_{\text{kinetic}}\rangle}{k_B} = \frac{1}{k_B}\left\langle\frac{1}{2}m v^2\right\rangle
\end{equation}
where $m$ is effective molecular mass, $v$ is velocity.
\end{equation}

\begin{theorem}[Temperature-Oscillation Correspondence]
\label{thm:temp_oscillation}
Genomic temperature relates to charge oscillation frequency:
\begin{equation}
k_BT = \hbar\omega_{\text{eff}}
\end{equation}
where $\omega_{\text{eff}}$ is effective oscillation frequency.
\end{theorem}

\begin{proof}
Equipartition theorem: $\langle E_{\text{kinetic}}\rangle = \frac{1}{2}k_BT$ per degree of freedom. For oscillator: $\langle E_{\text{kinetic}}\rangle = \frac{1}{2}\hbar\omega$. Equating:
\begin{equation}
\frac{1}{2}k_BT = \frac{1}{2}\hbar\omega \implies k_BT = \hbar\omega
\end{equation}
For genomic systems at $T = 310$ K (body temperature):
\begin{equation}
\omega_{\text{eff}} = \frac{k_BT}{\hbar} = \frac{1.38 \times 10^{-23} \times 310}{1.05 \times 10^{-34}} \approx 4 \times 10^{13} \text{ rad/s}
\end{equation}
corresponding to period $\tau \sim 10^{-13}$ s (molecular vibrations).
\end{proof}

\subsection{Pressure as Categorical Density}

\begin{definition}[Genomic Pressure]
\label{def:genomic_pressure}
Pressure quantifies categorical state density:
\begin{equation}
P_{\text{genomic}} = \frac{N}{V}k_BT
\end{equation}
where $N$ is number of categorical states, $V$ is nuclear volume.
\end{definition}

\begin{theorem}[Pressure-Partition Correspondence]
\label{thm:pressure_partition}
Genomic pressure relates to partition density:
\begin{equation}
P = \frac{1}{3}\rho\langle v^2\rangle = \frac{1}{3}\frac{N m}{V}\langle v^2\rangle
\end{equation}
where $\rho = Nm/V$ is mass density.
\end{theorem}

\begin{proof}
Kinetic theory: pressure arises from momentum transfer during collisions with boundaries. For $N$ particles in volume $V$, each with mass $m$ and velocity $v$:
\begin{equation}
P = \frac{1}{3}\frac{N m\langle v^2\rangle}{V}
\end{equation}
Using $\langle v^2\rangle = 3k_BT/m$ from equipartition:
\begin{equation}
P = \frac{1}{3}\frac{N m}{V} \cdot \frac{3k_BT}{m} = \frac{Nk_BT}{V}
\end{equation}
For genomic systems, "particles" are categorical gene states, "collisions" are partition crossings (gene state transitions).
\end{proof}

\subsection{Ideal Gas Law for Genomic Systems}

\begin{theorem}[Genomic Ideal Gas Law]
\label{thm:genomic_ideal_gas}
Genomic systems obey ideal gas law:
\begin{equation}
PV = Nk_BT
\end{equation}
where $P$ is categorical pressure, $V$ is nuclear volume, $N$ is number of gene states, $T$ is categorical temperature.
\end{theorem}

\begin{proof}
Combine pressure definition $P = Nk_BT/V$ with volume constraint to obtain $PV = Nk_BT$. For human cell:
\begin{itemize}
\item Nuclear volume: $V \sim 500$ $\mu$m$^3 = 5 \times 10^{-16}$ m$^3$
\item Number of genes: $N \sim 2 \times 10^4$
\item Temperature: $T = 310$ K
\end{itemize}
Predicted pressure:
\begin{equation}
P = \frac{Nk_BT}{V} = \frac{2 \times 10^4 \times 1.38 \times 10^{-23} \times 310}{5 \times 10^{-16}} \approx 1.7 \times 10^5 \text{ Pa}
\end{equation}
This matches measured nuclear pressure (osmotic pressure from chromatin), confirming ideal gas law applicability.
\end{proof}

\subsection{Maxwell-Boltzmann Distribution for Gene Expression}

\begin{theorem}[Gene Expression Distribution]
\label{thm:gene_expression_dist}
Gene expression levels follow Maxwell-Boltzmann distribution:
\begin{equation}
f(v) = \left(\frac{m}{2\pi k_BT}\right)^{3/2} 4\pi v^2 \exp\left(-\frac{mv^2}{2k_BT}\right)
\end{equation}
where $v$ represents expression rate.
\end{theorem}

\begin{proof}
In thermal equilibrium, velocities (expression rates) distribute according to Maxwell-Boltzmann. For genomic systems, "velocity" corresponds to transcription rate. Distribution peaks at:
\begin{equation}
v_{\text{peak}} = \sqrt{\frac{2k_BT}{m}}
\end{equation}
For effective mass $m \sim 10^6$ Da (transcription complex), temperature $T = 310$ K:
\begin{equation}
v_{\text{peak}} = \sqrt{\frac{2 \times 1.38 \times 10^{-23} \times 310}{10^6 \times 1.66 \times 10^{-27}}} \approx 150 \text{ m/s}
\end{equation}
Converting to transcription rate (base pairs per second):
\begin{equation}
r_{\text{transcribe}} = \frac{v_{\text{peak}}}{d_{\text{bp}}} = \frac{150}{0.34 \times 10^{-9}} \approx 4 \times 10^{11} \text{ bp/s}
\end{equation}
Actual transcription rate $\sim 50$ bp/s, indicating effective mass much larger than transcription complex alone (includes chromatin remodeling, regulatory factors).
\end{proof}

\subsection{Entropy Formulations}

\begin{theorem}[Three Forms of Genomic Entropy]
\label{thm:three_entropies}
Genomic entropy admits three equivalent formulations:
\begin{align}
S_{\text{oscillatory}} &= k_B\ln\Omega_{\text{phase}} \quad \text{(phase space volume)} \\
S_{\text{categorical}} &= -k_B\sum p_i\ln p_i \quad \text{(Shannon entropy)} \\
S_{\text{partition}} &= k_B\ln N_{\text{partitions}} \quad \text{(partition count)}
\end{align}
\end{theorem}

\begin{proof}
\textbf{Oscillatory entropy}: Boltzmann entropy $S = k_B\ln\Omega$ where $\Omega$ is number of accessible microstates. For oscillator in phase space volume $V_{\text{phase}}$:
\begin{equation}
\Omega_{\text{phase}} = \frac{V_{\text{phase}}}{h^3}
\end{equation}
where $h$ is Planck constant (quantum of phase space volume).

\textbf{Categorical entropy}: Shannon entropy for probability distribution $\{p_i\}$ over categorical states:
\begin{equation}
S_{\text{categorical}} = -k_B\sum_{i=1}^N p_i\ln p_i
\end{equation}
Maximized for uniform distribution: $S_{\text{max}} = k_B\ln N$.

\textbf{Partition entropy}: Entropy from partition count. For $N$ distinguishable partitions:
\begin{equation}
S_{\text{partition}} = k_B\ln N_{\text{partitions}}
\end{equation}

Triple equivalence establishes $\Omega_{\text{phase}} = N_{\text{categorical}} = N_{\text{partitions}}$, yielding identical entropies.
\end{proof}

\subsection{Genomic Equation of State}

\begin{theorem}[Genomic Equation of State]
\label{thm:genomic_eos}
Genomic systems obey equation of state:
\begin{equation}
\left(\frac{\partial U}{\partial V}\right)_T = T\left(\frac{\partial P}{\partial T}\right)_V - P
\end{equation}
where $U$ is internal energy (charge capacitance energy).
\end{theorem}

\begin{proof}
From thermodynamic identity $dU = TdS - PdV$, derive Maxwell relation:
\begin{equation}
\left(\frac{\partial U}{\partial V}\right)_T = T\left(\frac{\partial P}{\partial T}\right)_V - P
\end{equation}
For ideal gas, $P = Nk_BT/V$, so:
\begin{equation}
\left(\frac{\partial P}{\partial T}\right)_V = \frac{Nk_B}{V}
\end{equation}
Thus:
\begin{equation}
\left(\frac{\partial U}{\partial V}\right)_T = T\frac{Nk_B}{V} - \frac{Nk_BT}{V} = 0
\end{equation}
Internal energy independent of volume at constant temperature, characteristic of ideal gas. For genomic systems, this implies charge capacitance energy depends only on temperature (metabolic activity level), not nuclear volume.
\end{proof}

\subsection{Partition Function for Genomic States}

\begin{definition}[Genomic Partition Function]
\label{def:genomic_partition_function}
Partition function summing over all genomic states:
\begin{equation}
Z = \sum_{i=1}^N \exp\left(-\frac{E_i}{k_BT}\right)
\end{equation}
where $E_i$ is energy of state $i$ (gene expression configuration).
\end{definition}

\begin{theorem}[Thermodynamic Quantities from Partition Function]
\label{thm:thermo_from_partition}
All thermodynamic quantities derive from partition function:
\begin{align}
F &= -k_BT\ln Z \quad \text{(free energy)} \\
S &= k_B\left(\ln Z + T\frac{\partial\ln Z}{\partial T}\right) \quad \text{(entropy)} \\
\langle E\rangle &= -\frac{\partial\ln Z}{\partial\beta} \quad \text{(mean energy, } \beta = 1/k_BT\text{)}
\end{align}
\end{theorem}

\begin{proof}
Standard statistical mechanics derivations. Free energy:
\begin{equation}
F = -k_BT\ln Z = -k_BT\ln\left(\sum_i e^{-\beta E_i}\right)
\end{equation}
Entropy from $S = -(\partial F/\partial T)_V$:
\begin{equation}
S = -\frac{\partial}{\partial T}(-k_BT\ln Z) = k_B(\ln Z + T\frac{\partial\ln Z}{\partial T})
\end{equation}
Mean energy from $\langle E\rangle = -\partial\ln Z/\partial\beta$:
\begin{equation}
\langle E\rangle = \frac{\sum_i E_i e^{-\beta E_i}}{\sum_i e^{-\beta E_i}} = -\frac{1}{Z}\frac{\partial Z}{\partial\beta} = -\frac{\partial\ln Z}{\partial\beta}
\end{equation}
\end{proof}

\subsection{Genomic Heat Capacity}

\begin{theorem}[Genomic Heat Capacity]
\label{thm:genomic_heat_capacity}
Heat capacity at constant volume:
\begin{equation}
C_V = \left(\frac{\partial\langle E\rangle}{\partial T}\right)_V = \frac{k_B\beta^2\langle(\Delta E)^2\rangle}{k_BT} = k_B\beta^2\langle(\Delta E)^2\rangle
\end{equation}
where $\langle(\Delta E)^2\rangle = \langle E^2\rangle - \langle E\rangle^2$ is energy variance.
\end{theorem}

\begin{proof}
Heat capacity measures energy fluctuations:
\begin{equation}
C_V = \frac{\partial\langle E\rangle}{\partial T} = \frac{\partial}{\partial T}\left(-\frac{\partial\ln Z}{\partial\beta}\right)
\end{equation}
Using $\beta = 1/k_BT$ and chain rule:
\begin{equation}
C_V = k_B\beta^2\frac{\partial^2\ln Z}{\partial\beta^2} = k_B\beta^2\langle(\Delta E)^2\rangle
\end{equation}
For genomic systems, energy fluctuations arise from gene expression stochasticity. Large fluctuations (high $C_V$) indicate responsive system (stem cells), small fluctuations (low $C_V$) indicate stable system (differentiated cells).
\end{proof}

\subsection{Connection to Poincaré Computing}

\begin{theorem}[Ideal Gas Laws from Trajectory Completion]
\label{thm:ideal_gas_poincare}
Ideal gas laws derive equivalently from Poincaré trajectory completion:
\begin{equation}
PV = Nk_BT \iff \text{Trajectory completion rate} = \frac{1}{\tau_{\text{Poincare}}}
\end{equation}
\end{theorem}

\begin{proof}
Poincaré recurrence time $\tau_{\text{Poincare}} \sim V/v$ where $V$ is phase space volume, $v$ is velocity. Completion rate:
\begin{equation}
\frac{1}{\tau_{\text{Poincare}}} \sim \frac{v}{V}
\end{equation}
From kinetic theory, $v \sim \sqrt{k_BT/m}$. Pressure $P \sim Nmv^2/V \sim Nk_BT/V$. Thus:
\begin{equation}
PV \sim Nk_BT \iff \frac{1}{\tau_{\text{Poincare}}} \sim \frac{\sqrt{k_BT/m}}{V}
\end{equation}
Both formulations describe same physics: particle trajectories in bounded space. Ideal gas law emerges from trajectory completion statistics.
\end{proof}

This establishes that ideal gas laws for genomic systems derive from two independent starting points: (1) partition operations on bounded systems, (2) Poincaré trajectory completion. The equivalence confirms deep connection between thermodynamics and categorical computing.
