\section{Reset Dynamics and Cell Division Cycles}
\label{sec:division_cycles}

The preceding sections established that configuration space dilution suppresses pathological convergence in large systems. However, the analysis assumed ergodic sampling of configuration space. In reality, biological systems exhibit a crucial additional feature: periodic reset dynamics, where cellular configurations are partially restored toward reference states during cell division.

This section develops the mathematical theory of reset dynamics and demonstrates that cell division acts as an error-correction mechanism, further suppressing pathological trajectory accumulation. The key insight is that division interrupts the continuous accumulation of trajectory deviations, preventing runaway divergence from healthy configurations.

\subsection{Biological Motivation}

Cell division is the most fundamental reset mechanism in multicellular organisms. During mitosis:
\begin{enumerate}
    \item The genome is replicated and segregated, resetting epigenetic marks to varying degrees
    \item Cellular organelles are redistributed between daughter cells
    \item Accumulated damage products are diluted by a factor of two
    \item Protein aggregates and misfolded proteins are asymmetrically partitioned
    \item The cytoskeleton is disassembled and rebuilt
\end{enumerate}

This process is not a complete reset---daughter cells retain substantial information from their parent---but it introduces a controlled regression toward the cell type's characteristic state. The degree of inheritance versus reset varies by cell type and organism.

\begin{remark}[Cell Division Rates]
Cell division rates vary enormously across tissues:
\begin{itemize}
    \item Intestinal epithelium: $\tau \sim 1$--$5$ days
    \item Skin epidermis: $\tau \sim 2$--$4$ weeks
    \item Liver hepatocytes: $\tau \sim 200$--$300$ days (normally quiescent)
    \item Neurons: $\tau \to \infty$ (post-mitotic, non-dividing)
    \item Stem cells: Highly regulated, tissue-dependent
\end{itemize}
The variation in reset period has profound implications for error accumulation and cancer risk.
\end{remark}

\subsection{Hierarchical Reset Process}

We model the biological reset process through a discrete-time stochastic map with period $\tau$.

\begin{definition}[Reset Operator]
The reset operator $\mathcal{R}_\tau: \Config \to \Config$ maps a configuration $\mathbf{c}$ to a new configuration $\mathbf{c}'$ according to:
\begin{equation}
\mathbf{c}' = \mathcal{R}_\tau(\mathbf{c}) = (1 - \alpha) \mathbf{c}_0 + \alpha \mathbf{c} + \boldsymbol{\eta}
\end{equation}
where:
\begin{itemize}
    \item $\mathbf{c}_0$ is the reference (healthy) configuration for the cell type
    \item $\alpha \in [0, 1]$ is the inheritance parameter controlling state transmission
    \item $\boldsymbol{\eta}$ is a stochastic noise term with $\langle \boldsymbol{\eta} \rangle = 0$ and $\langle \eta_i \eta_j \rangle = \sigma^2 \delta_{ij}$
\end{itemize}
\end{definition}

The inheritance parameter $\alpha$ encodes the degree of epigenetic and cytoplasmic memory:
\begin{itemize}
    \item $\alpha = 0$: Complete reset to reference state (no memory)
    \item $\alpha = 1$: Complete inheritance with additive noise (perfect memory)
    \item $0 < \alpha < 1$: Partial inheritance with regression toward reference
\end{itemize}

\begin{remark}[Reference Configuration]
The reference configuration $\mathbf{c}_0$ is not a single fixed point but rather a statistical attractor---a distribution over healthy configurations characteristic of the cell type. The reset operator causes regression toward this distribution, not toward a unique state.
\end{remark}

\begin{proposition}[Reset as Contraction]
\label{prop:contraction}
For $\alpha < 1$, the reset operator is a contraction mapping on the space of deviations from the reference configuration:
\begin{equation}
\|\mathbf{c}' - \mathbf{c}_0\| \leq \alpha \|\mathbf{c} - \mathbf{c}_0\| + \|\boldsymbol{\eta}\|
\end{equation}
In expectation (ignoring noise):
\begin{equation}
\langle \|\mathbf{c}' - \mathbf{c}_0\| \rangle \leq \alpha \langle \|\mathbf{c} - \mathbf{c}_0\| \rangle
\end{equation}
\end{proposition}

\subsection{Deviation Dynamics Between Resets}

Define the deviation from the reference configuration:
\begin{equation}
\boldsymbol{\Delta}(t) = \mathbf{c}(t) - \mathbf{c}_0
\end{equation}

Between reset events, the deviation evolves according to the intrinsic dynamics of the coupled oscillator system. For a system with diffusive exploration of configuration space, the mean-squared deviation grows linearly with time.

\begin{definition}[Configuration Space Diffusion]
The diffusion coefficient $D$ in configuration space characterises the rate of random exploration:
\begin{equation}
\langle \Delta_i(t) \Delta_j(t) \rangle - \langle \Delta_i(t) \rangle \langle \Delta_j(t) \rangle = 2D t \delta_{ij} + \mathcal{O}(t^2)
\end{equation}
for short times $t \ll \tau_c$, where $\tau_c$ is the correlation time.
\end{definition}

The diffusion coefficient $D$ has dimensions of (configuration units)$^2$/time and depends on the intrinsic noise in cellular dynamics (thermal fluctuations, stochastic gene expression, metabolic noise).

\begin{figure}[htbp]
\centering
\includegraphics[width=0.9\textwidth]{figures/reset_dynamics.png}
\caption{\textbf{Reset dynamics and error dilution showing critical timescale scaling and regime transitions.}
\textbf{(Left, Panel G)} Critical timescale scaling with system size showing quarter-power law. Vertical axis is critical timescale $\tau_c$ (arbitrary units, logarithmic scale, $10^{-1}$ to $3 \times 10^{-1}$). Horizontal axis is number of oscillators $N$ (logarithmic scale, $10^2$ to $10^4$). Computed values (cyan circles connected by line) show $\tau_c$ decreasing with increasing $N$. Dashed gray line shows theoretical prediction $\tau_c \propto N^{-1/4}$ (Theorem~\ref{thm:correlation_time}). Excellent agreement between computed and theoretical values validates quarter-power scaling law across two orders of magnitude in $N$. For $N = 100$: $\tau_c \approx 3 \times 10^{-1}$. For $N = 10^4$: $\tau_c \approx 10^{-1}$ (threefold decrease over 100-fold increase in $N$). Scaling exponent extracted from log-log slope: $d(\log \tau_c)/d(\log N) \approx -0.25 \pm 0.02$, consistent with $-1/4$ prediction. Biological interpretation: critical timescale $\tau_c$ represents the correlation time of the slowest collective mode in hierarchical oscillator network (Section~\ref{sec:oscillatory_coherence}).
\textbf{(Right, Panel H)} Error dynamics regimes showing transition between error accumulation and error dilution. Vertical axis is mean squared deviation (logarithmic scale, $10^{-2}$ to $10^0$). Horizontal axis is generation number (0 to 50). Two traces: error accumulation regime (orange line, top) and error dilution regime (green oscillating line, bottom). Error accumulation regime (orange): mean squared deviation rises rapidly from $\sim 2$ at generation 0 to $\sim 3$ at generation 5, then plateaus at $\sim 3$ for generations 5--50. Plateau indicates saturation at maximum error level (system has converged to pathological attractor). Exponential growth phase (generations 0--5) reflects error propagation through configuration space without reset-induced decorrelation. Growth rate $\lambda = \Delta(\text{MSD})/\Delta t \sim 0.2$ per generation. Saturation level MSD $\sim 3$ represents typical distance from healthy attractor to pathological attractor in configuration space. Biological interpretation: error accumulation regime corresponds to $\tau < \tau^*$ (subcritical reset period). Cell divisions occur too frequently for configurations to decorrelate, allowing errors to propagate coherently across generations. }
\label{fig:reset_dynamics}
\end{figure}


\begin{proposition}[Diffusive Accumulation]
\label{prop:diffusion}
Between reset events at times $n\tau$ and $(n+1)\tau$, the deviation accumulates diffusively:
\begin{equation}
\langle |\boldsymbol{\Delta}((n+1)\tau^-)|^2 \rangle = \langle |\boldsymbol{\Delta}(n\tau^+)|^2 \rangle + 2D\tau \cdot N
\end{equation}
where $n\tau^+$ denotes immediately after the $n$-th reset and $(n+1)\tau^-$ denotes immediately before the $(n+1)$-th reset.
\end{proposition}

At each reset event (occurring at times $t = n\tau$ for $n \in \mathbb{Z}^+$), the deviation is transformed according to:
\begin{equation}
\boldsymbol{\Delta}^{(n+1)} = \alpha \boldsymbol{\Delta}^{(n)} + \boldsymbol{\xi}^{(n)}
\end{equation}
where $\boldsymbol{\xi}^{(n)}$ incorporates both the diffusive accumulation during the interval $[(n-1)\tau, n\tau]$ and the reset noise:
\begin{equation}
\langle \xi_i^{(n)} \xi_j^{(m)} \rangle = (2D\tau + \sigma^2) \delta_{nm} \delta_{ij} \equiv \Sigma^2 \delta_{nm} \delta_{ij}
\end{equation}

The effective noise variance $\Sigma^2 = 2D\tau + \sigma^2$ combines continuous diffusive accumulation ($2D\tau$) and discrete reset noise ($\sigma^2$).

\subsection{Stationary Deviation Distribution}

The competition between inheritance ($\alpha < 1$ causes regression) and noise ($\Sigma^2$ causes diffusion) determines the stationary distribution of deviations.

\begin{theorem}[Stationary Deviation Variance]
\label{thm:stationary_variance}
For the reset dynamics with inheritance parameter $\alpha < 1$ and effective noise $\Sigma^2$, the stationary variance of the deviation is:
\begin{equation}
\langle |\boldsymbol{\Delta}|^2 \rangle_{\infty} = \frac{\Sigma^2}{1 - \alpha^2}
\end{equation}
\end{theorem}

\begin{proof}
The variance after $n$ reset cycles satisfies the recurrence:
\begin{equation}
\langle |\boldsymbol{\Delta}^{(n+1)}|^2 \rangle = \langle |\alpha \boldsymbol{\Delta}^{(n)} + \boldsymbol{\xi}^{(n)}|^2 \rangle = \alpha^2 \langle |\boldsymbol{\Delta}^{(n)}|^2 \rangle + \Sigma^2
\end{equation}
where we used $\langle \boldsymbol{\Delta}^{(n)} \cdot \boldsymbol{\xi}^{(n)} \rangle = 0$ (the noise is independent of the prior state).

This is an autoregressive process AR(1) with coefficient $\alpha^2$ and innovation variance $\Sigma^2$. The stationary distribution exists if and only if $|\alpha| < 1$, and satisfies:
\begin{equation}
\langle |\boldsymbol{\Delta}|^2 \rangle_\infty = \alpha^2 \langle |\boldsymbol{\Delta}|^2 \rangle_\infty + \Sigma^2
\end{equation}
Solving: $\langle |\boldsymbol{\Delta}|^2 \rangle_\infty (1 - \alpha^2) = \Sigma^2$, hence:
\begin{equation}
\langle |\boldsymbol{\Delta}|^2 \rangle_\infty = \frac{\Sigma^2}{1 - \alpha^2} = \frac{2D\tau + \sigma^2}{1 - \alpha^2}
\end{equation}
$\square$
\end{proof}

\begin{corollary}[Bounds on Stationary Variance]
\label{cor:variance_bounds}
The stationary variance satisfies:
\begin{equation}
\Sigma^2 \leq \langle |\boldsymbol{\Delta}|^2 \rangle_\infty < \infty \quad \text{for} \quad \alpha < 1
\end{equation}
\begin{equation}
\lim_{\alpha \to 0} \langle |\boldsymbol{\Delta}|^2 \rangle_\infty = \Sigma^2 \quad (\text{single-cycle contribution})
\end{equation}
\begin{equation}
\lim_{\alpha \to 1^-} \langle |\boldsymbol{\Delta}|^2 \rangle_\infty = \infty \quad (\text{divergence})
\end{equation}
\end{corollary}

\subsection{Inheritance Parameter and Reset Period}

The inheritance parameter $\alpha$ depends on the reset period $\tau$ through the coupling dynamics. During the interval $[0, \tau]$, oscillators explore configuration space, with exploration range proportional to $\sqrt{D\tau}$.

\begin{theorem}[Effective Inheritance]
\label{thm:inheritance}
For diffusive dynamics with configuration space diffusion coefficient $D$ and intrinsic correlation time $\tau_c$, the effective inheritance parameter is:
\begin{equation}
\alpha(\tau) = e^{-\tau/\tau_c}
\end{equation}
\end{theorem}

\begin{proof}
The inheritance parameter measures the normalised autocorrelation of configurations at times separated by $\tau$:
\begin{equation}
\alpha(\tau) = \frac{\langle \mathbf{c}(t) \cdot \mathbf{c}(t+\tau) \rangle - |\langle \mathbf{c} \rangle|^2}{\langle |\mathbf{c}|^2 \rangle - |\langle \mathbf{c} \rangle|^2}
\end{equation}
For an Ornstein-Uhlenbeck process with correlation time $\tau_c$, the autocorrelation decays exponentially: $\alpha(\tau) = e^{-\tau/\tau_c}$. This is the generic form for overdamped dynamics in a confining potential \citep{gardiner2009stochastic}.

The Ornstein-Uhlenbeck process satisfies:
\begin{equation}
d\mathbf{c} = -\frac{1}{\tau_c}(\mathbf{c} - \mathbf{c}_0) \, dt + \sqrt{2D} \, d\mathbf{W}
\end{equation}
where $\mathbf{W}$ is a Wiener process. The solution has autocorrelation:
\begin{equation}
\langle (\mathbf{c}(t) - \langle \mathbf{c} \rangle) \cdot (\mathbf{c}(t+\tau) - \langle \mathbf{c} \rangle) \rangle = D\tau_c \cdot e^{-|\tau|/\tau_c}
\end{equation}
Normalising by the variance $D\tau_c$ gives $\alpha(\tau) = e^{-|\tau|/\tau_c}$. $\square$
\end{proof}

\begin{corollary}[Stationary Variance as Function of Reset Period]
\label{cor:variance_tau}
Combining Theorem~\ref{thm:stationary_variance} and Theorem~\ref{thm:inheritance}:
\begin{equation}
\langle |\boldsymbol{\Delta}|^2 \rangle_\infty (\tau) = \frac{2D\tau + \sigma^2}{1 - e^{-2\tau/\tau_c}}
\end{equation}
\end{corollary}

\subsection{Critical Timescale Analysis}

The dependence of stationary variance on reset period reveals two distinct regimes.

\begin{proposition}[Short Reset Period Regime]
\label{prop:short_tau}
For $\tau \ll \tau_c$, expanding $e^{-2\tau/\tau_c} \approx 1 - 2\tau/\tau_c$:
\begin{equation}
\langle |\boldsymbol{\Delta}|^2 \rangle_\infty \approx \frac{2D\tau + \sigma^2}{2\tau/\tau_c} = \frac{(2D\tau + \sigma^2)\tau_c}{2\tau}
\end{equation}
For very small $\tau$:
\begin{equation}
\langle |\boldsymbol{\Delta}|^2 \rangle_\infty \approx D\tau_c + \frac{\sigma^2 \tau_c}{2\tau} \sim \frac{1}{\tau} \quad (\tau \to 0)
\end{equation}
\end{proposition}

The divergence as $\tau \to 0$ reflects the fact that very frequent resets, each introducing noise $\sigma^2$, accumulate into large variance. The reset noise dominates over the error-correction benefit.

\begin{proposition}[Long Reset Period Regime]
\label{prop:long_tau}
For $\tau \gg \tau_c$, the inheritance parameter $\alpha = e^{-\tau/\tau_c} \ll 1$, and:
\begin{equation}
\langle |\boldsymbol{\Delta}|^2 \rangle_\infty \approx 2D\tau + \sigma^2 \approx 2D\tau
\end{equation}
\end{proposition}

In this regime, each reset nearly completely erases the prior state, and the variance is dominated by diffusive accumulation during the single inter-reset interval. The variance grows linearly with $\tau$.

\begin{theorem}[Optimal Reset Period]
\label{thm:optimal_reset}
The stationary variance $\langle |\boldsymbol{\Delta}|^2 \rangle_\infty (\tau)$ has a minimum at an optimal reset period $\tau^*$ satisfying:
\begin{equation}
\tau^* = \mathcal{O}(\tau_c)
\end{equation}
Specifically, for $\sigma^2 \ll D\tau_c$ (diffusion-dominated noise):
\begin{equation}
\tau^* \approx 0.35 \tau_c
\end{equation}
\end{theorem}

\begin{proof}
The variance function:
\begin{equation}
V(\tau) = \frac{2D\tau + \sigma^2}{1 - e^{-2\tau/\tau_c}}
\end{equation}
has derivative:
\begin{equation}
\frac{dV}{d\tau} = \frac{2D(1 - e^{-2\tau/\tau_c}) - (2D\tau + \sigma^2) \cdot \frac{2}{\tau_c} e^{-2\tau/\tau_c}}{(1 - e^{-2\tau/\tau_c})^2}
\end{equation}
Setting the numerator to zero and defining $x = 2\tau/\tau_c$:
\begin{equation}
2D(1 - e^{-x}) = (D\tau_c x + \sigma^2) \cdot \frac{2}{\tau_c} e^{-x}
\end{equation}
For $\sigma^2 \ll D\tau_c$, this simplifies to:
\begin{equation}
1 - e^{-x} = x e^{-x}
\end{equation}
which has solution $x \approx 0.70$, giving $\tau^* \approx 0.35 \tau_c$. $\square$
\end{proof}

\begin{definition}[Critical Reset Period]
The critical reset period $\tau^*$ is defined by the condition $\alpha(\tau^*) = 1/\sqrt{2}$:
\begin{equation}
\tau^* = \frac{\tau_c}{2} \ln 2 \approx 0.347 \tau_c
\end{equation}
\end{definition}

This definition is equivalent to the optimal reset period in the diffusion-dominated regime.

\begin{theorem}[Error Accumulation Criterion]
\label{thm:accumulation}
\begin{enumerate}
    \item For $\tau > \tau^*$: Deviations regress toward the population mean between resets. Pathological configurations are dissipated.
    \item For $\tau < \tau^*$: Deviations accumulate across reset cycles. Pathological configurations can propagate.
\end{enumerate}
\end{theorem}

\subsection{Population-Level Implications}

The reset dynamics described above apply to individual cells. For a population of $N$ cells, the collective effect depends on the correlation structure.

\begin{proposition}[Independent Cell Resets]
\label{prop:independent_resets}
If cells reset independently (asynchronous division), the population variance is:
\begin{equation}
\langle |\boldsymbol{\Delta}_{\text{pop}}|^2 \rangle = \frac{1}{N} \langle |\boldsymbol{\Delta}|^2 \rangle_\infty
\end{equation}
where $\boldsymbol{\Delta}_{\text{pop}} = \frac{1}{N} \sum_i \boldsymbol{\Delta}_i$ is the population-averaged deviation.
\end{proposition}

\begin{proof}
For independent cells:
\begin{equation}
\langle |\boldsymbol{\Delta}_{\text{pop}}|^2 \rangle = \frac{1}{N^2} \sum_{i,j} \langle \boldsymbol{\Delta}_i \cdot \boldsymbol{\Delta}_j \rangle = \frac{1}{N^2} \sum_i \langle |\boldsymbol{\Delta}_i|^2 \rangle = \frac{1}{N} \langle |\boldsymbol{\Delta}|^2 \rangle_\infty
\end{equation}
using $\langle \boldsymbol{\Delta}_i \cdot \boldsymbol{\Delta}_j \rangle = 0$ for $i \neq j$ (independence). $\square$
\end{proof}

\begin{corollary}[Population-Level Error Suppression]
The population-level variance decreases as $1/N$, providing an additional suppression mechanism beyond configuration space dilution:
\begin{equation}
\langle |\boldsymbol{\Delta}_{\text{pop}}|^2 \rangle \propto \frac{1}{N} \cdot \frac{\Sigma^2}{1 - \alpha^2}
\end{equation}
\end{corollary}

\subsection{Hierarchical Reset Structure}

Biological systems exhibit reset dynamics at multiple scales:
\begin{enumerate}
    \item \textbf{Molecular scale}: Protein turnover ($\tau \sim$ hours to days)
    \item \textbf{Cellular scale}: Cell division ($\tau \sim$ days to weeks)
    \item \textbf{Tissue scale}: Tissue renewal ($\tau \sim$ weeks to months)
    \item \textbf{Circadian scale}: Daily rhythms ($\tau = 24$ hours)
\end{enumerate}

\begin{theorem}[Hierarchical Reset Suppression]
\label{thm:hierarchical_reset}
For a system with $L$ hierarchical levels of reset, with reset periods $\tau_1 < \tau_2 < \cdots < \tau_L$, the stationary variance satisfies:
\begin{equation}
\langle |\boldsymbol{\Delta}|^2 \rangle_\infty \leq \prod_{\ell=1}^{L} \frac{1}{1 - \alpha_\ell^2} \cdot \Sigma_1^2
\end{equation}
where $\alpha_\ell = e^{-\tau_\ell / \tau_{\ell+1}}$ and $\Sigma_1^2$ is the noise at the fastest level.
\end{theorem}

The multiplicative structure implies that hierarchical resets provide compound error suppression.

\begin{figure}[htbp]
\centering
\includegraphics[width=\textwidth]{figures/figure2_okazaki_fragments.png}
\caption{\textbf{Okazaki Fragment Length: Charge-Dependent Replication Dynamics.} 
\textbf{(A)} DNA replication progression. Cumulative replicated DNA (bp) increases linearly with time for three conditions: eukaryote with oscillating [Mg$^{2+}$] (blue solid line), eukaryote with constant [Mg$^{2+}$] (red dashed line), and prokaryote with high [Mg$^{2+}$] (green dash-dot line). Eukaryotic replication reaches $\sim$15,000 bp at 300 s, while prokaryotic replication reaches $\sim$22,000 bp (47\% faster). 
\textbf{(B)} Eukaryote fragment length (oscillating [Mg$^{2+}$]). Histogram shows bimodal distribution with peaks at 150 nt and 154 nt (blue bars). Red dashed line: mean = 151.2 nt. 
\textbf{(C)} Eukaryote fragment length (constant [Mg$^{2+}$]). Histogram shows sharp unimodal distribution at 149.7 nt (red bars). Blue dashed line: mean = 149.7 nt. 
\textbf{(D)} Prokaryote fragment length (high [Mg$^{2+}$]). Histogram shows narrow distribution centered at 1596.0 nt (green bars). Red dashed line: mean = 1596.0 nt.  
\textbf{(E)} Okazaki fragment length oscillations over time. Fragment length (nt) oscillates between 150 nt and 155 nt with period $\sim$5 s (blue line), synchronized with ATP synthesis cycle. 
\textbf{(F)} Fragment length vs. [Mg$^{2+}$]. Fragment length (nt) increases linearly with [Mg$^{2+}$] concentration from 149 nt at 0.0 mM to 163 nt at 2.0 mM (purple line). Blue dashed line: eukaryote baseline ([Mg$^{2+}$] = 0.3 mM, fragment length = 151 nt).  
\textbf{(G)} Fragment length comparison. Bar chart compares mean fragment lengths: eukaryote (oscillating) = 151 $\pm$ 2 nt (blue bar), eukaryote (constant) = 150 $\pm$ 0 nt (red bar), prokaryote = 1596 $\pm$ 0 nt (green bar). Prokaryote/eukaryote ratio = 10.6$\times$, matching predicted $\sim$10$\times$ difference from literature. 
\textbf{(H)} Fragment length variability. Coefficient of variation (CV, \%) quantifies relative variability: eukaryote (oscillating) = 1.5\% (blue bar), eukaryote (constant) = 0.0\% (gray bar), prokaryote = 0.0\% (gray bar).}
\label{fig:okazaki_fragments}
\end{figure}

\subsection{Connection to Cancer Biology}

The reset dynamics framework provides insight into why some tissues are more cancer-prone than others.

\begin{proposition}[Tissue-Specific Cancer Risk]
\label{prop:tissue_risk}
Tissues with:
\begin{enumerate}
    \item Longer cell cycle times ($\tau$ larger) have higher single-cycle deviation
    \item Lower cell turnover (fewer resets per lifetime) have less error correction
    \item Post-mitotic cells ($\tau \to \infty$) accumulate deviations without bound
\end{enumerate}
\end{proposition}

\begin{example}[Colon vs.\ Brain Cancer]
Colon epithelial cells divide every $\sim 4$ days, providing frequent reset opportunities. Despite the high proliferation rate, the short reset period ($\tau \sim 4$ days $\ll \tau_c$) maintains bounded deviation.

Neurones are post-mitotic ($\tau \to \infty$) and cannot reset through division. However, neurones have very low metabolic noise (stable configurations) and strong inter-neuronal coupling that provide error correction through network effects. Primary brain tumours are rare; most brain cancers are metastatic from other sites.
\end{example}

\subsection{Summary}

The reset dynamics of cell division provide a second line of defence against pathological trajectory accumulation:
\begin{enumerate}
    \item Deviations accumulated between divisions are partially erased at each reset
    \item The stationary deviation variance is bounded for $\alpha < 1$
    \item An optimal reset period $\tau^* \sim 0.35 \tau_c$ minimises deviation variance
    \item Hierarchical reset at multiple scales provides compound suppression
\end{enumerate}

Combined with configuration space dilution (Section~\ref{sec:configuration_space}) and spatial decorrelation (Section~\ref{sec:spatial_decorrelation}), reset dynamics explain why large organisms are not overwhelmed by pathological events despite their astronomical cell counts.

