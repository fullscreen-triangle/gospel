\section{Error Propagation and System Size Scaling}
\label{sec:error_propagation}

The preceding sections developed the individual components of Peto's paradox resolution: configuration space dilution, spatial decorrelation, reset dynamics, and bounded trajectory deviations. This section synthesises these results into the main theorem on error rate scaling, demonstrating rigorously that the pathological convergence probability decreases with system size despite the increasing cell count.

The key insight is that errors must not only occur but must also \emph{propagate} through reset events and \emph{coordinate} across spatially separated regions to produce system-level pathology. Each of these requirements introduces exponential suppression factors that dominate the linear increase in cell count.

\subsection{Error Definition and Classification}

We begin with a precise definition of what constitutes an error in the oscillator network framework.

\begin{definition}[Local Error]
A local error at oscillator $i$ is a deviation $\delta_i = \sigma_i - \sigma_i^{(0)}$ from the reference state $\sigma_i^{(0)}$ such that $\delta_i \in \mathcal{E}$, where $\mathcal{E} \subset \mathbb{R}$ is the error set.
\end{definition}

The error set $\mathcal{E}$ characterises deviations that, if accumulated or propagated, could contribute to pathological configurations. Different error sets correspond to different pathological mechanisms:

\begin{example}[Error Set Types]
\begin{enumerate}
    \item \textbf{Threshold errors}: $\mathcal{E} = \{|\delta| > \delta_{\text{th}}\}$ --- deviations exceeding a critical magnitude
    \item \textbf{Signed errors}: $\mathcal{E} = \{\delta > \delta_{\text{th}}\}$ --- deviations in a specific direction (e.g., oncogenic gain-of-function)
    \item \textbf{Interval errors}: $\mathcal{E} = \{\delta \in [\delta_{\min}, \delta_{\max}]\}$ --- deviations within a specific range (e.g., loss of regulation)
\end{enumerate}
\end{example}

\begin{definition}[Error Event]
An error event at oscillator $i$ and reset cycle $n$ is the event $\{\delta_i^{(n)} \in \mathcal{E}\}$.
\end{definition}

\begin{definition}[Local Error Rate]
The local error rate $p_{\text{local}}$ is the probability of an error event per oscillator per reset cycle in the stationary regime:
\begin{equation}
p_{\text{local}} = \Pr[\delta_i^{(n)} \in \mathcal{E}]
\end{equation}
under the stationary distribution of the reset dynamics.
\end{definition}

\subsection{Error Propagation Through Resets}

A crucial question is whether errors persist through reset events. The reset dynamics of Section~\ref{sec:division_cycles} provide the answer.

\begin{definition}[Error Propagation]
An error propagates from cycle $n$ to cycle $n+1$ if:
\begin{equation}
\delta_i^{(n)} \in \mathcal{E} \quad \text{and} \quad \delta_i^{(n+1)} \in \mathcal{E}
\end{equation}
The propagation probability is:
\begin{equation}
P_{\text{prop}} = \Pr[\delta_i^{(n+1)} \in \mathcal{E} \,|\, \delta_i^{(n)} \in \mathcal{E}]
\end{equation}
\end{definition}

From the reset dynamics:
\begin{equation}
\delta_i^{(n+1)} = \alpha \delta_i^{(n)} + \xi_i^{(n)}
\end{equation}
where $\alpha = e^{-\tau/\tau_c}$ is the inheritance parameter, and $\xi_i^{(n)} \sim \mathcal{N}(0, \Sigma^2)$ is the reset noise.

\begin{theorem}[Conditional Distribution After Reset]
\label{thm:conditional_distribution}
Given $\delta_i^{(n)} = \delta_0$, the post-reset deviation is:
\begin{equation}
\delta_i^{(n+1)} \,|\, \delta_i^{(n)} = \delta_0 \sim \mathcal{N}(\alpha \delta_0, \Sigma^2)
\end{equation}
\end{theorem}

\begin{proof}
The direct consequence of the linearity of the reset map is the independence of $\xi_i^{(n)}$ from $\delta_i^{(n)}$. $\square$
\end{proof}

\begin{theorem}[Propagation Probability for Threshold Errors]
\label{thm:propagation_probability}
For threshold errors $\mathcal{E} = \{|\delta| > \delta_{\text{th}}\}$ and initial error $\delta_i^{(n)} = \delta_0$ with $|\delta_0| > \delta_{\text{th}}$:
\begin{equation}
P_{\text{prop}}(\delta_0) = 1 - \Phi\left(\frac{\delta_{\text{th}} - \alpha \delta_0}{\Sigma}\right) + \Phi\left(\frac{-\delta_{\text{th}} - \alpha \delta_0}{\Sigma}\right)
\end{equation}
where $\Phi(x) = \frac{1}{\sqrt{2\pi}} \int_{-\infty}^{x} e^{-t^2/2} dt$ is the standard normal CDF.
\end{theorem}

\begin{proof}
The event $\{|\delta_i^{(n+1)}| > \delta_{\text{th}}\}$ is the union $\{\delta_i^{(n+1)} > \delta_{\text{th}}\} \cup \{\delta_i^{(n+1)} < -\delta_{\text{th}}\}$. These events are disjoint, so:
\begin{align}
P_{\text{prop}} &= \Pr[\delta_i^{(n+1)} > \delta_{\text{th}}] + \Pr[\delta_i^{(n+1)} < -\delta_{\text{th}}] \\
&= 1 - \Pr[\delta_i^{(n+1)} \leq \delta_{\text{th}}] + \Pr[\delta_i^{(n+1)} < -\delta_{\text{th}}]
\end{align}
Standardising: $Z = (\delta_i^{(n+1)} - \alpha\delta_0)/\Sigma \sim \mathcal{N}(0,1)$, so:
\begin{align}
\Pr[\delta_i^{(n+1)} \leq \delta_{\text{th}}] &= \Pr[Z \leq (\delta_{\text{th}} - \alpha\delta_0)/\Sigma] = \Phi((\delta_{\text{th}} - \alpha\delta_0)/\Sigma) \\
\Pr[\delta_i^{(n+1)} < -\delta_{\text{th}}] &= \Pr[Z < (-\delta_{\text{th}} - \alpha\delta_0)/\Sigma] = \Phi((-\delta_{\text{th}} - \alpha\delta_0)/\Sigma)
\end{align}
$\square$
\end{proof}

\begin{lemma}[Propagation Decay]
\label{lemma:propagation_decay}
For $\alpha < 1$ and minimal initial error $\delta_0 = \delta_{\text{th}}$, the propagation probability satisfies:
\begin{equation}
P_{\text{prop}} = 1 - \Phi\left(\frac{(1-\alpha)\delta_{\text{th}}}{\Sigma}\right) + \Phi\left(\frac{-(1+\alpha)\delta_{\text{th}}}{\Sigma}\right)
\end{equation}
In the limit $(1-\alpha)\delta_{\text{th}} / \Sigma \gg 1$:
\begin{equation}
P_{\text{prop}} \sim \frac{\Sigma}{(1-\alpha)\delta_{\text{th}} \sqrt{2\pi}} \exp\left(-\frac{(1-\alpha)^2 \delta_{\text{th}}^2}{2\Sigma^2}\right)
\end{equation}
\end{lemma}

\begin{proof}
Substitute $\delta_0 = \delta_{\text{th}}$ into Theorem~\ref{thm:propagation_probability}. For the asymptotic behaviour, use the Gaussian tail bound:
\begin{equation}
1 - \Phi(x) \sim \frac{1}{x\sqrt{2\pi}} e^{-x^2/2} \quad \text{as } x \to \infty
\end{equation}
The second term $\Phi(-(1+\alpha)\delta_{\text{th}}/\Sigma)$ is exponentially smaller than the first (since $(1+\alpha) > (1-\alpha)$ for $\alpha > 0$) and can be neglected. $\square$
\end{proof}

\begin{corollary}[Exponential Suppression of Propagation]
For $\alpha < 1$ and sufficiently large $\delta_{\text{th}}/\Sigma$:
\begin{equation}
P_{\text{prop}} \lesssim e^{-(1-\alpha)^2 \delta_{\text{th}}^2 / (2\Sigma^2)}
\end{equation}
The propagation probability is exponentially suppressed by the ratio $(\delta_{\text{th}}/\Sigma)^2$.
\end{corollary}

\begin{figure}[htbp]
\centering
\includegraphics[width=\textwidth]{figures/validation_panel.png}
\caption{.
\textbf{(Top left, Panel A)} Configuration space explosion (identical to Figure~\ref{fig:petos_paradox}, Panel A). Logarithmic configuration space size $\log_{10}|\Config|$ versus cell count $N$ showing exponential scaling $|\Config| = \Omega^N$ (cyan curve with markers). Four organisms marked: mouse, human, elephant, whale. 
\textbf{(Top right, Panel B)} Synchronization dynamics showing time evolution of Kuramoto order parameter $r(t)$ for three coupling strengths: $K = 0.5$ (magenta), $K = 1.0$ (orange), $K = 2.0$ (orange, highest curve). Horizontal dashed line at $r = 0.9$ marks synchronization threshold. Strong coupling ($K = 2.0$) achieves $r \sim 0.83$ by $t = 10$ s (approaching synchronization). Intermediate coupling ($K = 1.0$) shows transient synchronization. Weak coupling ($K = 0.5$) remains incoherent. Validates Kuramoto model predictions (Proposition~\ref{prop:phase_lock}): critical coupling $K_c \sim 1$ separates incoherent ($K < K_c$) from synchronized ($K > K_c$) phases.
\textbf{(Middle left, Panel C)} Coupling dependence showing final synchronization $r_{\infty}$ versus coupling strength $K$. Sigmoidal transition from $r \sim 0.05$ at $K = 0$ to $r \sim 0.98$ at $K = 5$. Critical coupling $K_c \approx 1$ (vertical orange dashed line) marks inflection point.
\textbf{(Middle center, Panel D)} Size dependence at fixed coupling ($K = 1$) showing final synchronization $r_{\infty}$ versus number of oscillators $N$. Non-monotonic behavior: $r$ increases from $\sim 0.05$ at $N = 10$ to $\sim 0.22$ at $N = 100$, then decreases to $\sim 0.10$ at $N = 300$. Peak at $N \sim 100$ represents optimal balance between mean-field stabilization (favors large $N$) and frequency heterogeneity (favors small $N$). 
\textbf{(Middle right, Panel E)} Configuration space versus oscillators showing $\log|\Config|$ versus $N$ (cyan curve with markers). Linear relationship on log-linear plot confirms exponential scaling $|\Config| = \Omega^N$. Dashed line shows theoretical prediction $N \ln(\Omega)$ with excellent agreement. 
\textbf{(Bottom left, Panel F)} Configuration space versus states per oscillator showing $\log|\Config|$ versus $\Omega$ (magenta curve with markers). Linear relationship on log-linear plot confirms $|\Config| = \Omega^N$ with $N$ fixed. 
\textbf{(Bottom center-left, Panel G)} Critical timescale scaling (identical to Figure~\ref{fig:reset_dynamics}, Panel G). Shows $\tau_c$ versus $N$ with quarter-power scaling $\tau_c \propto N^{-1/4}$ or $N^{1/4}$ (depending on definition). Computed values (cyan circles) match theoretical prediction (gray dashed line) across two orders of magnitude in $N$. 
\textbf{(Bottom center-right, Panel H)} Error dynamics regimes (identical to Figure~\ref{fig:reset_dynamics}, Panel H). Shows mean squared deviation versus generation for error accumulation (orange, constant high level) and error dilution (green, oscillating low level) regimes. 
\textbf{(Bottom left, Panel I)} Hierarchical frequency levels (identical to Figure~\ref{fig:hierarchical_coupling}, Panel I). Horizontal bar chart showing $\log_{10}(\text{frequency})$ for nine biological processes spanning 20 orders of magnitude: quantum coherence ($\sim 10^{15}$ Hz) to environmental coupling ($\sim 10^{-5}$ Hz).
\textbf{(Bottom right, Panel J)} Multi-scale synchronization (identical to Figure~\ref{fig:hierarchical_coupling}, Panel J). Time series of order parameter $r(t)$ for four hierarchical scales: enzyme catalysis (orange), synaptic transmission (purple), action potentials (blue), circadian rhythms (gray), plus global coherence (black).(Proposition~\ref{prop:phase_lock}).
 }
\label{fig:validation_panel}
\end{figure}

\subsection{Multi-Generation Error Dynamics}

We now analyse how errors evolve over multiple reset generations.

\begin{theorem}[Multi-Generation Error Evolution]
\label{thm:multi_generation}
After $n$ reset generations, an initial error $\delta_0$ evolves to:
\begin{equation}
\delta^{(n)} = \alpha^n \delta_0 + \sum_{k=0}^{n-1} \alpha^k \xi^{(n-1-k)}
\end{equation}
where $\{\xi^{(m)}\}$ are i.i.d.\ $\mathcal{N}(0, \Sigma^2)$ noise terms.
\end{theorem}

\begin{proof}
Iteration of the reset map $\delta^{(n+1)} = \alpha \delta^{(n)} + \xi^{(n)}$:
\begin{align}
\delta^{(1)} &= \alpha \delta_0 + \xi^{(0)} \\
\delta^{(2)} &= \alpha(\alpha \delta_0 + \xi^{(0)}) + \xi^{(1)} = \alpha^2 \delta_0 + \alpha \xi^{(0)} + \xi^{(1)} \\
&\vdots \\
\delta^{(n)} &= \alpha^n \delta_0 + \sum_{k=0}^{n-1} \alpha^k \xi^{(n-1-k)}
\end{align}
$\square$
\end{proof}

\begin{corollary}[Mean and Variance Evolution]
The mean and variance of $\delta^{(n)}$ are:
\begin{align}
\langle \delta^{(n)} \rangle &= \alpha^n \delta_0 \\
\text{Var}[\delta^{(n)}] &= \Sigma^2 \sum_{k=0}^{n-1} \alpha^{2k} = \Sigma^2 \frac{1 - \alpha^{2n}}{1 - \alpha^2}
\end{align}
\end{corollary}

\begin{theorem}[Error Regression]
\label{thm:error_regression}
For $\alpha < 1$, an initial error of magnitude $\delta_0$ regresses to the stationary distribution within a characteristic number of generations:
\begin{equation}
n^* = \frac{\ln(\delta_0 / \sigma_\infty)}{|\ln \alpha|} = \frac{\tau_c}{\tau} \ln\left(\frac{\delta_0}{\sigma_\infty}\right)
\end{equation}
where $\sigma_\infty = \Sigma / \sqrt{1-\alpha^2}$ is the stationary standard deviation.
\end{theorem}

\begin{proof}
The mean deviation satisfies $|\langle \delta^{(n)} \rangle| = \alpha^n |\delta_0|$. Setting this equal to the stationary standard deviation:
\begin{equation}
\alpha^{n^*} |\delta_0| = \sigma_\infty
\end{equation}
Taking logarithms:
\begin{equation}
n^* \ln \alpha = \ln(\sigma_\infty / |\delta_0|)
\end{equation}
Since $\ln \alpha = -\tau/\tau_c < 0$:
\begin{equation}
n^* = \frac{\ln(|\delta_0| / \sigma_\infty)}{\tau/\tau_c} = \frac{\tau_c}{\tau} \ln\left(\frac{|\delta_0|}{\sigma_\infty}\right)
\end{equation}
$\square$
\end{proof}

\begin{corollary}[Error Memory Time]
The time required for an error to regress to the stationary distribution is:
\begin{equation}
T_{\text{memory}} = n^* \cdot \tau = \tau_c \ln\left(\frac{|\delta_0|}{\sigma_\infty}\right)
\end{equation}
This depends only on $\tau_c$ and the initial error magnitude, not on the reset period $\tau$.
\end{corollary}

\subsection{Error Coordination Across Spatial Domains}

Pathological configurations typically require coordination of errors across multiple cells within and across correlation volumes.

\begin{definition}[Coordinated Error]
A coordinated error event requires errors at a specific subset $\mathcal{S} \subset \{1, \ldots, N\}$ of oscillators:
\begin{equation}
\text{Coordinated error} = \bigcap_{i \in \mathcal{S}} \{\delta_i \in \mathcal{E}\}
\end{equation}
\end{definition}

\begin{theorem}[Coordination Probability]
\label{thm:coordination}
Let $\mathcal{S}$ span $k$ distinct correlation volumes. The probability of a coordinated error event is:
\begin{equation}
P_{\text{coord}}(\mathcal{S}) = \prod_{\alpha=1}^{k} P_{\text{local error in } V_\alpha}
\end{equation}
where $V_\alpha$ denotes the $\alpha$-th correlation volume.
\end{theorem}

\begin{proof}
By Theorem~\ref{thm:factorisation} (configuration space factorisation), oscillators in distinct correlation volumes are statistically independent. The joint probability factorises as:
\begin{equation}
\Pr\left[\bigcap_{i \in \mathcal{S}} \{\delta_i \in \mathcal{E}\}\right] = \prod_{\alpha=1}^{k} \Pr\left[\bigcap_{i \in \mathcal{S} \cap V_\alpha} \{\delta_i \in \mathcal{E}\}\right]
\end{equation}
$\square$
\end{proof}

\begin{corollary}[Multiplicative Suppression]
For a coordinated error requiring $m_\alpha$ oscillators in volume $V_\alpha$ to be in error states simultaneously:
\begin{equation}
P_{\text{coord}} = \prod_{\alpha=1}^{k} P_{\text{local}}(m_\alpha, V_\alpha)
\end{equation}
If each local probability is $p < 1$, then:
\begin{equation}
P_{\text{coord}} = p^k \xrightarrow{k \to \infty} 0
\end{equation}
The coordination probability vanishes exponentially with the number of required correlation volumes.
\end{corollary}

\subsection{Global Error Rate}

We now define and analyse the global error rate---the rate at which pathological configurations arise system-wide.

\begin{definition}[Global Error Rate]
The global error rate $\Gamma$ is the expected number of pathological events per unit time:
\begin{equation}
\Gamma = \frac{1}{\tau} \cdot N \cdot p_{\text{local}} \cdot P_{\text{prop}} \cdot P_{\text{coord}}
\end{equation}
where:
\begin{itemize}
    \item $\tau^{-1}$ = reset frequency (division rate)
    \item $N$ = number of oscillators (cells)
    \item $p_{\text{local}}$ = local error probability per oscillator per cycle
    \item $P_{\text{prop}}$ = propagation probability through reset
    \item $P_{\text{coord}}$ = coordination probability across required spatial extent
\end{itemize}
\end{definition}

\begin{remark}[Interpretation of Terms]
Each factor in the global error rate has a distinct physical interpretation:
\begin{itemize}
    \item $N / \tau$: Rate of ``attempts'' (cell divisions) system-wide
    \item $p_{\text{local}}$: Fraction of attempts generating local errors
    \item $P_{\text{prop}}$: Fraction of local errors surviving reset
    \item $P_{\text{coord}}$: Fraction of propagated errors achieving coordination
\end{itemize}
\end{remark}

\subsection{Main Theorem: Error Rate Scaling}

\begin{theorem}[Error Rate Scaling]
\label{thm:main}
For a system of $N$ oscillators with $\Omega$ states each, the correlation length $\xi$, reset period $\tau$, and correlation time $\tau_c$ satisfy the global error rate: 
\begin{equation}
\Gamma = \frac{N \cdot p_{\text{local}}}{\tau} \cdot \frac{1}{\Omega^{n_\xi}} \cdot f\left(\frac{\tau}{\tau_c}\right)
\end{equation}
where:
\begin{itemize}
    \item $n_\xi = N \cdot V_\xi / V$ is the number of oscillators per correlation volume
    \item $f(x)$ is a monotonically decreasing function with:
\end{itemize}
\begin{equation}
f(x) = \begin{cases}
\mathcal{O}(1) & x \ll 1 \text{ (frequent reset)} \\
\mathcal{O}(e^{-cx}) & x \gg 1 \text{ (infrequent reset)}
\end{cases}
\end{equation}
for some constant $c > 0$ depending on the error threshold.
\end{theorem}

\begin{proof}
We construct the proof by combining results from previous sections.

\textbf{Step 1: Coordination probability.}
By Theorem~\ref{thm:coordination} and Section~\ref{sec:configuration_space}, the probability of a pathological configuration requiring coherence within a single correlation volume is:
\begin{equation}
P_{\text{coord}} \leq \frac{|\Attr_{\text{local}}|}{|\Config_{\text{local}}|} = \frac{|\Attr_{\text{local}}|}{\Omega^{n_\xi}}
\end{equation}
For pathological subset with $|\Attr_{\text{local}}| = \mathcal{O}(1)$ (a specific configuration type):
\begin{equation}
P_{\text{coord}} \sim \frac{1}{\Omega^{n_\xi}}
\end{equation}

\textbf{Step 2: Propagation probability.}
By Lemma~\ref{lemma:propagation_decay}, the propagation probability is:
\begin{equation}
P_{\text{prop}} \sim \exp\left(-\frac{(1-\alpha)^2 \delta_{\text{th}}^2}{2\Sigma^2}\right)
\end{equation}
Substituting $\alpha = e^{-\tau/\tau_c}$:
\begin{equation}
1 - \alpha = 1 - e^{-\tau/\tau_c} \approx \frac{\tau}{\tau_c} \text{ for } \tau \ll \tau_c
\end{equation}
\begin{equation}
1 - \alpha \approx 1 \text{ for } \tau \gg \tau_c
\end{equation}
Define:
\begin{equation}
g(x) = \frac{(1 - e^{-x})^2 \delta_{\text{th}}^2}{2\Sigma^2}
\end{equation}
which is monotonically increasing in $x = \tau/\tau_c$. Then $P_{\text{prop}} = e^{-g(\tau/\tau_c)}$.

\textbf{Step 3: Combined rate.}
\begin{equation}
\Gamma = \frac{N \cdot p_{\text{local}}}{\tau} \cdot P_{\text{prop}} \cdot P_{\text{coord}} = \frac{N \cdot p_{\text{local}}}{\tau} \cdot e^{-g(\tau/\tau_c)} \cdot \frac{1}{\Omega^{n_\xi}}
\end{equation}
Defining $f(x) = e^{-g(x)}$, which is monotonically decreasing:
\begin{equation}
\Gamma = \frac{N \cdot p_{\text{local}}}{\tau} \cdot \frac{1}{\Omega^{n_\xi}} \cdot f\left(\frac{\tau}{\tau_c}\right)
\end{equation}

\textbf{Step 4: Asymptotic behaviour of $f$.}
For $x \ll 1$: $g(x) \approx x^2 \delta_{\text{th}}^2 / (2\Sigma^2) \ll 1$, so $f(x) \approx 1$.
For $x \gg 1$: $g(x) \approx \delta_{\text{th}}^2 / (2\Sigma^2) = \text{const.}$, so $f(x) \approx e^{-\delta_{\text{th}}^2/(2\Sigma^2)} = \text{const.}$ but the ratio $\tau/\tau_c$ in the denominator of $\Gamma$ provides additional suppression. More precisely, incorporating all $\tau$-dependent terms:
\begin{equation}
\frac{f(\tau/\tau_c)}{\tau} \sim \frac{e^{-c\tau/\tau_c}}{\tau}
\end{equation}
for large $\tau$. $\square$
\end{proof}

\subsection{Comparison Across System Sizes}

\begin{theorem}[Size-Invariance of Error Rate]
\label{thm:invariance}
Consider two systems with cell counts $N_1$ and $N_2 = \lambda N_1$ for $\lambda > 1$, maintaining the same tissue type (same $\Omega$, $\xi$, $\tau$, $\tau_c$, $p_{\text{local}}$). The ratio of global error rates is:
\begin{equation}
\frac{\Gamma_2}{\Gamma_1} = \lambda \cdot \Omega^{-(n_{\xi,2} - n_{\xi,1})}
\end{equation}
\end{theorem}

\begin{proof}
From Theorem~\ref{thm:main}:
\begin{equation}
\frac{\Gamma_2}{\Gamma_1} = \frac{N_2}{N_1} \cdot \frac{\Omega^{n_{\xi,1}}}{\Omega^{n_{\xi,2}}} = \lambda \cdot \Omega^{n_{\xi,1} - n_{\xi,2}}
\end{equation}
If the larger system has proportionally more cells per correlation volume (e.g., larger cells), then $n_{\xi,2} = \lambda n_{\xi,1}$, yielding:
\begin{equation}
\frac{\Gamma_2}{\Gamma_1} = \lambda \cdot \Omega^{-(\lambda - 1) n_{\xi,1}}
\end{equation}
$\square$
\end{proof}

\begin{corollary}[Exponential Suppression in Large Systems]
For $\Omega > 1$ and $n_{\xi,1} \geq 1$:
\begin{equation}
\frac{\Gamma_2}{\Gamma_1} < 1 \quad \text{for all } \lambda > 1 + \frac{\ln \lambda}{n_{\xi,1} \ln \Omega}
\end{equation}
For typical values ($\Omega \sim 10^5$, $n_{\xi,1} \sim 10^3$), this condition is satisfied for $\lambda > 1 + 10^{-8} \ln \lambda \approx 1$ --- effectively all $\lambda > 1$.
\end{corollary}

\begin{example}[Quantitative Comparison]
For $N_1 = 10^{10}$ (mouse), $N_2 = 10^{15}$ (whale), $\Omega = 10^5$, $n_{\xi,1} = 10^3$:
\begin{equation}
\lambda = 10^5, \quad n_{\xi,2} - n_{\xi,1} \approx (10^5 - 1) \cdot 10^3 \approx 10^8
\end{equation}
\begin{equation}
\frac{\Gamma_{\text{whale}}}{\Gamma_{\text{mouse}}} = 10^5 \cdot (10^5)^{-10^8} = 10^5 \cdot 10^{-5 \times 10^8} = 10^{5 - 5 \times 10^8} \approx 10^{-5 \times 10^8}
\end{equation}
This ratio is so small that it is effectively zero. The whale's error rate is suppressed by a factor of $10^{500\text{ million}}$ relative to the naive expectation.
\end{example}

\begin{figure}[htbp]
\centering
\includegraphics[width=\textwidth]{figures/petos_paradox.png}
\caption{\textbf{Peto's paradox resolution through configuration space explosion showing exponential suppression of pathology probability with system size.}
\textbf{(Left, Panel A)} Configuration space explosion showing exponential growth of state space with cell count. Main plot: Vertical axis is logarithmic configuration space size $\log_{10}|\Config|$ (base-10 logarithm of total number of configurations). Horizontal axis is cell count $N$ (logarithmic scale, $10^6$ to $10^{15}$). Curve (cyan with circular markers) shows linear relationship on log-log plot: $\log_{10}|\Config| \propto N$, indicating exponential scaling $|\Config| = \Omega^N$ (Proposition~\ref{prop:dilution}). Four organisms marked with vertical dashed lines: mouse ($N \sim 10^9$, $\log_{10}|\Config| \sim 10^{12}$), human ($N \sim 10^{13}$, $\log_{10}|\Config| \sim 10^{17}$), elephant ($N \sim 10^{14}$, $\log_{10}|\Config| \sim 10^{18}$), whale ($N \sim 10^{15}$, $\log_{10}|\Config| \sim 10^{19}$).
\textbf{(Right, Organism Comparison)} Quantitative comparison of cell count and configuration space across four organisms. Vertical axis is $\log_{10}$ value. Horizontal axis lists four organisms: mouse, human, elephant, whale. Two bar series: cell count (cyan bars) and configuration space size (magenta bars). Cell count increases from mouse ($\log_{10} N \sim 9$) to whale ($\log_{10} N \sim 16$), spanning 7 orders of magnitude. Configuration space increases from mouse ($\log_{10}|\Config| \sim 13$) to whale ($\log_{10}|\Config| \sim 20$), spanning 7 orders of magnitude. Crucially, configuration space bars are consistently $\sim 4$--5 orders of magnitude higher than cell count bars, reflecting exponential scaling $|\Config| = \Omega^N$ with $\Omega \sim 10^{4-5}$ (number of states per cell). }
\label{fig:petos_paradox}
\end{figure}

\subsection{Invariance Condition and Optimal System Size}

\begin{theorem}[Optimal System Size]
\label{thm:optimal_size}
The global error rate $\Gamma(N)$ as a function of system size has a unique maximum at:
\begin{equation}
N^* = \frac{V}{V_\xi \ln \Omega}
\end{equation}
For $N < N^*$, the error rate increases with $N$. For $N > N^*$, the error rate decreases with $N$.
\end{theorem}

\begin{proof}
From Theorem~\ref{thm:main}, ignoring the $f(\tau/\tau_c)$ factor (which depends on $\tau$, not $N$):
\begin{equation}
\Gamma(N) \propto N \cdot \Omega^{-n_\xi} = N \cdot \Omega^{-N \cdot V_\xi / V} = N \cdot e^{-N \cdot (V_\xi / V) \ln \Omega}
\end{equation}
Let $c = (V_\xi / V) \ln \Omega$. Then:
\begin{equation}
\Gamma(N) \propto N \cdot e^{-cN}
\end{equation}
Taking the derivative:
\begin{equation}
\frac{d\Gamma}{dN} \propto e^{-cN} - cN e^{-cN} = e^{-cN}(1 - cN)
\end{equation}
Setting to zero: $1 - cN^* = 0$, so $N^* = 1/c = V / (V_\xi \ln \Omega)$. $\square$
\end{proof}

\begin{remark}[Biological Interpretation]
The optimal size $N^* = V / (V_\xi \ln \Omega)$ represents the crossover between two regimes:
\begin{itemize}
    \item $N < N^*$: More cells provide more opportunities for errors; rate increases
    \item $N > N^*$: Configuration space dilution dominates; the rate decreases
\end{itemize}
For typical biological parameters, $N^* \sim 10^6$--$10^8$, well below multicellular organism sizes.
\end{remark}

\subsection{Summary and Implications}

The error propagation analysis establishes:
\begin{enumerate}
    \item \textbf{Propagation suppression}: Errors must survive reset events, with a probability exponentially small in $(1-\alpha)\delta_{\text{th}}/\Sigma$
    \item \textbf{Coordination suppression}: Errors must coordinate across correlation volumes, with a probability exponentially small in the number of volumes
    \item \textbf{Size scaling}: The combined effect is that the error rate \emph{decreases} with system size for $N > N^*$
    \item \textbf{Quantitative magnitude}: The suppression is not marginal but astronomical --- factors of $10^{10^8}$ for whale vs.\ mouse
\end{enumerate}

This provides the mathematical resolution of Peto's paradox: large organisms are not protected by better defences or slower metabolism but by the statistical mechanics of high-dimensional configuration spaces. The exponential growth of possible configurations with cell count dilutes the probability of any specific pathological configuration, overwhelming the linear increase in ``opportunities'' for pathology.

